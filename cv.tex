%-----------------------------------------------------------------------------------------------------------------------------------------------%
%	The MIT License (MIT)
%
%	Copyright (c) 2021 Jitin Nair
%
%	Permission is hereby granted, free of charge, to any person obtaining a copy
%	of this software and associated documentation files (the "Software"), to deal
%	in the Software without restriction, including without limitation the rights
%	to use, copy, modify, merge, publish, distribute, sublicense, and/or sell
%	copies of the Software, and to permit persons to whom the Software is
%	furnished to do so, subject to the following conditions:
%	
%	THE SOFTWARE IS PROVIDED "AS IS", WITHOUT WARRANTY OF ANY KIND, EXPRESS OR
%	IMPLIED, INCLUDING BUT NOT LIMITED TO THE WARRANTIES OF MERCHANTABILITY,
%	FITNESS FOR A PARTICULAR PURPOSE AND NONINFRINGEMENT. IN NO EVENT SHALL THE
%	AUTHORS OR COPYRIGHT HOLDERS BE LIABLE FOR ANY CLAIM, DAMAGES OR OTHER
%	LIABILITY, WHETHER IN AN ACTION OF CONTRACT, TORT OR OTHERWISE, ARISING FROM,
%	OUT OF OR IN CONNECTION WITH THE SOFTWARE OR THE USE OR OTHER DEALINGS IN
%	THE SOFTWARE.
%	
%
%-----------------------------------------------------------------------------------------------------------------------------------------------%

%----------------------------------------------------------------------------------------
%	DOCUMENT DEFINITION
%----------------------------------------------------------------------------------------

% article class because we want to fully customize the page and not use a cv template
\documentclass[a4paper,12pt]{article}

%----------------------------------------------------------------------------------------
%	FONT
%----------------------------------------------------------------------------------------

% % fontspec allows you to use TTF/OTF fonts directly
% \usepackage{fontspec}
% \defaultfontfeatures{Ligatures=TeX}

% % modified for ShareLaTeX use
% \setmainfont[
% SmallCapsFont = Fontin-SmallCaps.otf,
% BoldFont = Fontin-Bold.otf,
% ItalicFont = Fontin-Italic.otf
% ]
% {Fontin.otf}

%----------------------------------------------------------------------------------------
%	PACKAGES
%----------------------------------------------------------------------------------------
\usepackage{url}
\usepackage{parskip} 	

%other packages for formatting
\RequirePackage{color}
\RequirePackage{graphicx}
\usepackage[usenames,dvipsnames]{xcolor}
\usepackage[scale=0.9]{geometry}

%tabularx environment
\usepackage{tabularx}

%for lists within experience section
\usepackage{enumitem}

% centered version of 'X' col. type
\newcolumntype{C}{>{\centering\arraybackslash}X} 

%to prevent spillover of tabular into next pages
\usepackage{supertabular}
\usepackage{tabularx}
\newlength{\fullcollw}
\setlength{\fullcollw}{0.47\textwidth}

%custom \section
\usepackage{titlesec}				
\usepackage{multicol}
\usepackage{multirow}

%CV Sections inspired by: 
%http://stefano.italians.nl/archives/26
\titleformat{\section}{\Large\scshape\raggedright}{}{0em}{}[\titlerule]
\titlespacing{\section}{0pt}{10pt}{10pt}

%for publications
\usepackage[style=authoryear,sorting=ynt, maxbibnames=2]{biblatex}

%Setup hyperref package, and colours for links
\usepackage[unicode, draft=false]{hyperref}
\definecolor{linkcolour}{rgb}{0,0.2,0.6}
\hypersetup{colorlinks,breaklinks,urlcolor=linkcolour,linkcolor=linkcolour}
\addbibresource{citations.bib}
\setlength\bibitemsep{1em}

%for social icons
\usepackage{fontawesome5}

%debug page outer frames
%\usepackage{showframe}

%----------------------------------------------------------------------------------------
%	BEGIN DOCUMENT
%----------------------------------------------------------------------------------------
\begin{document}

% non-numbered pages
\pagestyle{empty} 

%----------------------------------------------------------------------------------------
%	TITLE
%----------------------------------------------------------------------------------------

% \begin{tabularx}{\linewidth}{ @{}X X@{} }
% \huge{Your Name}\vspace{2pt} & \hfill \emoji{incoming-envelope} email@email.com \\
% \raisebox{-0.05\height}\faGithub\ username \ | \
% \raisebox{-0.00\height}\faLinkedin\ username \ | \ \raisebox{-0.05\height}\faGlobe \ mysite.com  & \hfill \emoji{calling} number
% \end{tabularx}

\begin{tabularx}{\linewidth}{@{} C @{}}
\Huge{Ahmed Tidiane BALDE} \\[7.5pt]
\href{https://github.com/ATidiane}{\raisebox{-0.05\height}\faGithub\ atidiane} \ $|$ \ 
\href{https://linkedin.com/in/ahmed-balde}{\raisebox{-0.05\height}\faLinkedin\ ahmed-balde} \ $|$ \ 
\href{https://mysite.com}{\raisebox{-0.05\height}\faGlobe \ mysite.com} \ $|$ \ 
\href{mailto:ahmedt.balde@gmail.com}{\raisebox{-0.05\height}\faEnvelope \ ahmedt.balde@gmail.com} \ $|$ \ 
\href{tel:+33 6 69 78 36 06}{\raisebox{-0.05\height}\faMobile \ +33 6.69.78.36.06} \\
\end{tabularx}

%----------------------------------------------------------------------------------------
% EXPERIENCE SECTIONS
%----------------------------------------------------------------------------------------

%Interests/ Keywords/ Summary
\section{Summary}
This CV is automatically generated and deployed using the \href{https://github.com/jitinnair1/autoCV}{autoCV} template along with GitHub Actions such that a new version of the CV is compiled, published and ready for use when the cv.tex file is updated. For details, \href{https://github.com/jitinnair1/autoCV}{click here}.

%Experience
\section{Work Experience}

\begin{tabularx}{\linewidth}{ @{}l r@{} }
\textbf{Designation} & \hfill Jan 2021 - present \\[3.75pt]
\multicolumn{2}{@{}X@{}}{long long line of blah blah that will wrap when the table fills the column width long long line of blah blah that will wrap when the table fills the column width long long line of blah blah that will wrap when the table fills the column width long long line of blah blah that will wrap when the table fills the column width}  \\
\end{tabularx}

\begin{tabularx}{\linewidth}{ @{}l r@{} }
\textbf{R\&D Deep Learning Engineer Intern} & \hfill Mars 2020 - Août 2020 \\[3.75pt]
\textit{\textsc{Groupe PSA - Velizy Villacoublay}}
\multicolumn{2}{@{}X@{}}{
\begin{minipage}[t]{\linewidth}
    \begin{itemize}[nosep,after=\strut, leftmargin=1em, itemsep=3pt]
        \item[--] long long line of blah blah that will wrap when the table fills the column width
        \item[--] again, long long line of blah blah that will wrap when the table fills the column width but this time even more long long line of blah blah. again, long long line of blah blah that will wrap when the table fills the column width but this time even more long long line of blah blah
    \end{itemize}
    \end{minipage}
}
\end{tabularx}

%Projects
\section{Projets}

\begin{tabularx}{\linewidth}{ @{}l r@{} }
\textbf{Convolutional Neural Networks} & \hfill \href{https://github.com/ATidiane/RDFIA/tree/master/TMEs/tp6-7}{voir sur GitHub} \\
[3.75pt]
\multicolumn{2}{@{}X@{}}{Implémentation d’un réseau de neurones convolutifs dont l’architecture est proche d’AlexNet, pour la classification d’images sur la base CIFAR-10} \\

\textbf{Image Description with SIFT, bag of words  and SVM Classification} & \hfill \href{https://github.com/ATidiane/RDFIA/tree/master/TMEs/tp1-2-3}{voir sur GitHub} \\
[3.75pt]
\multicolumn{2}{@{}X@{}}{Dans le cadre d’un cours sur la reconnaissance des formes et l’interprétration d’images, nous avons été introduit au SIFT (Scale Invariant Feature Transform), qui est un descripteur d’image robuste à la rotation, la variation d’intensité ainsi qu’au zoom. Après avoir calculé les descripteurs de nos images distinctes, nous avons appliqué un K-Means avec un k défini à 1001 pour trouver les centres des clusters qui représentent au mieux les images. Le dernier cluster étant un vecteur de zéros pour les descripteurs vides. Puis, nous assignâmes à chacun des clusters, sa région la plus proche. Enfin, pour la tâche de classification, nous avons généré des caractéristiques pour chacune des images en fonction de notre dictionnaire visuel des régions.} \\

\textbf{Inpainting} & \hfill \href{https://github.com/ATidiane/ARF/tree/master/InpaintingProject}{voir sur GitHub} \\
[3.75pt]
\multicolumn{2}{@{}X@{}}{Dans un premier temps nous avons comparé la performance des algorithmes tels que la régression linéaire et la régression ridge sur les données USPS(United states Postal Service) pour la reconnaissance des chiffres et dans un second temps nous avons implémenté un ensemble de fonctions utilisant l’algorithme du Lasso pour prédire les pixels manquants d’une image. En un mot débruiter une image ou compléter une partie d’image manquante.} \\

% Example
%\textbf{Some Project} & \hfill \href{https://some-link.com}{Link to Demo} \\[3.75pt]
%\multicolumn{2}{@{}X@{}}{long long line of blah blah that will wrap when the table fills the column width long long line of blah blah that will wrap when the table fills the column width long long line of blah blah that will wrap when the table fills the column width long long line of blah blah that will wrap when the table fills the column width}  \\
\end{tabularx}

%----------------------------------------------------------------------------------------
%	EDUCATION
%----------------------------------------------------------------------------------------
\section{Formation}
\begin{tabularx}{\linewidth}{@{}l X@{}}	
2017 - 2020 & Master DAC (Données, Apprentissage et Connaissance) à \textbf{Sorbonne Université Sciences} \hfill \normalsize (Mention Bien) \\

2015 - 2017 & Licence Mono-Informatique à l'\textbf{Université de Pierre et Marie Curie (UPMC)} \hfill (Mention Assez Bien) \\ 

2014-2015 & Licence 1 Mathématiques-Informatique à l'Université Claude Bernard de Lyon1 (UCBL)\\

2014 & Baccalauréat Scientifique au Lycée Sainte Marie en Guinée \\
\end{tabularx}

%----------------------------------------------------------------------------------------
%	PUBLICATIONS
%----------------------------------------------------------------------------------------
\section{Publications}
\begin{refsection}[citations.bib]
\nocite{*}
\printbibliography[heading=none]
\end{refsection}

%----------------------------------------------------------------------------------------
%	SKILLS
%----------------------------------------------------------------------------------------
\section{Skills}
\begin{tabularx}{\linewidth}{@{}l X@{}}
Some Skills &  \normalsize{This, That, Some of this and that etc.}\\
Some More Skills  &  \normalsize{Also some more of this, Some more that, And some of this and that etc.}\\  
\end{tabularx}

\vfill
\center{\footnotesize Last updated: \today}

\end{document}
